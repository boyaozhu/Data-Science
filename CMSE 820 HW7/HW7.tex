\documentclass{article}

\usepackage{fancyhdr}
\usepackage{extramarks}
\usepackage{amsmath}
\usepackage{amsthm}
\usepackage{amsfonts}
\usepackage{tikz}
\usepackage[plain]{algorithm}
\usepackage{algpseudocode}
\usepackage{graphicx}
\usetikzlibrary{automata,positioning}
\usepackage{pdfpages}

%
% Basic Document Settings
%

\topmargin=-0.45in
\evensidemargin=0in
\oddsidemargin=0in
\textwidth=6.5in
\textheight=9.0in
\headsep=0.25in

\linespread{1.1}

\pagestyle{fancy}
\lhead{\hmwkAuthorName}
\chead{\hmwkClass\ (\hmwkClassInstructor\ \hmwkClassTime): \hmwkTitle}
\rhead{\firstxmark}
\lfoot{\lastxmark}
\cfoot{\thepage}

\renewcommand\headrulewidth{0.4pt}
\renewcommand\footrulewidth{0.4pt}

\setlength\parindent{0pt}

%
% Create Problem Sections
%

\newcommand{\enterProblemHeader}[1]{
    \nobreak\extramarks{}{Problem \arabic{#1} continued on next page\ldots}\nobreak{}
    \nobreak\extramarks{Problem \arabic{#1} (continued)}{Problem \arabic{#1} continued on next page\ldots}\nobreak{}
}

\newcommand{\exitProblemHeader}[1]{
    \nobreak\extramarks{Problem \arabic{#1} (continued)}{Problem \arabic{#1} continued on next page\ldots}\nobreak{}
    \stepcounter{#1}
    \nobreak\extramarks{Problem \arabic{#1}}{}\nobreak{}
}

\setcounter{secnumdepth}{0}
\newcounter{partCounter}
\newcounter{homeworkProblemCounter}
\setcounter{homeworkProblemCounter}{1}
\nobreak\extramarks{Problem \arabic{homeworkProblemCounter}}{}\nobreak{}

%
% Homework Problem Environment
%
% This environment takes an optional argument. When given, it will adjust the
% problem counter. This is useful for when the problems given for your
% assignment aren't sequential. See the last 3 problems of this template for an
% example.
%
\newenvironment{homeworkProblem}[1][-1]{
    \ifnum#1>0
        \setcounter{homeworkProblemCounter}{#1}
    \fi
    \section{Problem \arabic{homeworkProblemCounter}}
    \setcounter{partCounter}{1}
    \enterProblemHeader{homeworkProblemCounter}
}{
    \exitProblemHeader{homeworkProblemCounter}
}

%
% Homework Details
%   - Title
%   - Due date
%   - Class
%   - Section/Time
%   - Instructor
%   - Author
%

\newcommand{\hmwkTitle}{Homework\ \#7}
\newcommand{\hmwkDueDate}{Nov 3, 2019}
\newcommand{\hmwkClass}{CMSE 820}
\newcommand{\hmwkClassTime}{}
\newcommand{\hmwkClassInstructor}{Professor Yuying Xie}
\newcommand{\hmwkAuthorName}{\textbf{Boyao Zhu}}

%
% Title Page
%

\title{
    \vspace{2in}
    \textmd{\textbf{\hmwkClass:\ \hmwkTitle}}\\
    \normalsize\vspace{0.1in}\small{Due\ on\ \hmwkDueDate\ at 11:59pm}\\
    \vspace{0.1in}\large{\textit{\hmwkClassInstructor\ \hmwkClassTime}}
    \vspace{3in}
}

\author{\hmwkAuthorName}
\date{}

\renewcommand{\part}[1]{\textbf{\large Part \Alph{partCounter}}\stepcounter{partCounter}\\}

%
% Various Helper Commands
%

% Useful for algorithms
\newcommand{\alg}[1]{\textsc{\bfseries \footnotesize #1}}

% For derivatives
\newcommand{\deriv}[1]{\frac{\mathrm{d}}{\mathrm{d}x} (#1)}

% For partial derivatives
\newcommand{\pderiv}[2]{\frac{\partial}{\partial #1} (#2)}

% Integral dx
\newcommand{\dx}{\mathrm{d}x}

% Alias for the Solution section header
\newcommand{\solution}{\textbf{\large Solution}}

% Probability commands: Expectation, Variance, Covariance, Bias
\newcommand{\E}{\mathrm{E}}
\newcommand{\Var}{\mathrm{Var}}
\newcommand{\Cov}{\mathrm{Cov}}
\newcommand{\Bias}{\mathrm{Bias}}

\begin{document}

\maketitle

\pagebreak

\begin{homeworkProblem}
\textbf{Solution}\\

Let \(k_1(\cdot,\cdot)\) and \(k_2(\cdot,\cdot)\) be two reproducing kernels associated with an arbitrary RKHS \(\mathcal{H}\) over \(\mathcal{X}\).  Note that \(\forall x\in \mathcal{X}, k_1(\cdot,x)\in \mathcal{H}\) and \(k_2(\cdot,x)\in\mathcal{H}\).  Moreover, by the reproducing property, we have
\[
\forall x,y\in \mathcal{X}, \quad k_1(x,y)=k_1(y,x)=\langle k_1(\cdot,x),k_2(\cdot,y)\rangle_{\mathcal{H}}=\langle k_2(\cdot,y),k_1(\cdot,x)\rangle_{\mathcal{H}} = k_2(x,y).
\]
So the reproducing kernel is unique.
\end{homeworkProblem}


\begin{homeworkProblem}
\textbf{Solution}\\

\(\forall N\in \mathbb{N}^+, \forall x_1, x_2, \cdot\cdot\cdot, x_N\in \mathcal{X}, \forall c_1, c_2, \cdot\cdot\cdot, c_N\in \mathbb{R}\)
\[
\underset{i,j}{\huge\sum}c_ic_jf(x_i)k(x_i,x_j)f(x_j) = \underset{i,j}{\huge\sum}[c_if(x_i)][c_jf(x_j)]k(x_i,x_j)\geq 0,
\]
which implies immediately the positive semi-definiteness of \(\tilde{k}(x,y)\). (The last inequality follows from the positive semi-definiteness of \(k(x,y)\).
\end{homeworkProblem}


\begin{homeworkProblem}
\textbf{Solution}\\

Rewrite the kernel function as 
\[
k(x,y) = \text{min}\{x,y\}=\int_0^1\textsf{1}_{[0,x]}(z)\textsf{1}_{[0,y]}(z)dz.
\]
\(\forall N\in\mathbb{N}^+, \forall x_1,x_2, \cdot\cdot\cdot, x_N\in[0,1], \forall c_1, c_2, \cdot\cdot\cdot, c_N\in\mathbb{R}\),
\[
\underset{i,j}{\huge\sum}c_ic_j\text{min}\{x_i, x_j\}=\underset{i,j}{\huge\sum}\int_0^1 [c_i\textsf{1}_{[0,x_i]}(z)][c_j\textsf{1}_{[0,x_j]}(z)]dz = \int_0^1[\underset{i}{\huge\sum}c_i\textsf{1}_{[0,x_i]}(z)]^2dz\geq 0.
\]
So \(k\) is positive semi-definite.
\end{homeworkProblem}



\begin{homeworkProblem}
\textbf{Proof (1)}\\

\(\forall x\in[0,1]\), we can define a function \(g_x(z) = \text{min}\{x, z\} = 
\begin{cases} 
z, & z\leq x \\ 
x, & z>x
\end{cases}\).  It is obvious that \(g_x'(z) = 
\begin{cases}
1, & z<x \\
0, & z>x
\end{cases}\) is bounded and continuous almost everywhere (except at \(z = x\)), and hence integrable on [0, 1], which shows the absolute continuity of \(g_x\).  Moreover, \(g_x(0) = 0\) so \(g_x\in \mathcal{H}^1\).\\
Now consider the inner product \(\langle f, g_x\rangle_{\mathcal{H}^1}\) for any arbitrary \(x\in[0,1]\) and any \(f\in\mathcal{H}^1\).
\[
\langle f, g_x\rangle_{\mathcal{H}^1} = \int_0^x f'(z)\cdot 1dz + \int_x^1f'(z)\cdot 0 dz = \int_0^x f'(z)dz = f(x)
\]

\textbf{Proof (2)}\\

The equation above allows us to rewrite the evaluational functional as \(\delta_x(f)=f(x) = \langle f, g_x\rangle_{\mathcal{H}^1}\).  We'll show that \(\delta_x\) is bounded.
\[
\forall f\in\mathcal{H}^1, \lvert\delta_x(f)\rvert = \lvert\langle(f, g_x\rangle_{\mathcal{H}^1}\rvert\leq \lVert g_x\rVert_{\mathcal{H}^1}\lVert f\rVert_{\mathcal{H}^1}
\]
The operator norm of \(\delta_x\)
\[
\lVert \delta_x\rVert \leq \lVert g_x \rVert_{\mathcal{H}^1}=(\langle g_x, g_x\rangle_{\mathcal{H}^1})^{\frac{1}{2}}=\sqrt{x}\leq 1.
\]
Thus, \(\delta_x\) is bounded and \(mathcal{H}^1\) is an RKHS.
\end{homeworkProblem}










\end{document}